\documentclass{article}
\usepackage{amssymb}
\usepackage{amsmath}
\usepackage{epsfig}
\usepackage{subfigure}

\setlength{\textwidth}{160mm}
\setlength{\textheight}{255mm}
\setlength{\topmargin}{-15mm}
\setlength{\oddsidemargin}{0mm}
\setlength{\evensidemargin}{0mm}

\begin{document}

\title{Difference in the {\it uv}-plane?}
\author{TKP Software Group}
\maketitle

\section{Description}
Only point sources
The goal we want to achieve is to compare two sets of visibilities taken at subsequent times. 
For this we want to write the next observation in the {\it uv}-coverage of the previous.
For the next observation the {\it uv}-coverage is rotated - we know how it rotates - and this causes in the ideal case of only point sources in the field a phase shift of the visibilities. This shift is given by the difference in the projected baselines:

\begin{equation}
V_{ij}(t_{1}) = V_{ij}(t_{2}) e^{-2 \pi i ({\bf b_{ij}}(t_{1}) - {\bf b_{ij}}(t_{2})) \cdot {\bf \sigma} / c}
\end{equation}
where $t_{1}$ and $t_{2}$ are consecutive visibilities for the baseline between antennae $i$ and $j$.

\section{Visibilities}
For a point source the visibility (for one baseline) can be written as:
\begin{equation}
V(u) = A_{0} e^{-2 \pi i u l_{0}}
\end{equation}
and for a collection of point sources:

\begin{equation}
V(u) = \sum_{j} A_{j} e^{-2 \pi i u l_{j}}
\end{equation}
and further for a collection of point sources and multiple baselines:

\begin{equation}
V(u_{k}) = \sum_{j} A_{j} e^{-2 \pi i u_{k} l_{j}}
\end{equation}

\end{document}
