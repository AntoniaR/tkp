\documentclass{article}
\usepackage{amssymb}
\usepackage{amsmath}
\usepackage{epsfig}
\usepackage{subfigure}

\setlength{\textwidth}{160mm}
\setlength{\textheight}{255mm}
\setlength{\topmargin}{-15mm}
\setlength{\oddsidemargin}{0mm}
\setlength{\evensidemargin}{0mm}

\begin{document}

\title{Can we difference in the {\it uv}-plane?}
\author{James Miller-Jones}
\maketitle

Up until this point, we have assumed that the easiest way to detect
transients would be to difference two consecutive data sets in the
image plane.  Following discussions with Jan Noordam, it was decided
to investigate the possibility of differencing in the {\it uv}-plane
prior to imaging, in order that imperfectly-calibrated data might
cancel out, rather than creating artifacts in the image plane.

\section{Fourier transforms}
The image plane and {\it uv}-plane are related by a Fourier transform
relationship.  In order to simplify the discussion, we reduce the
problem to one dimension.  Let us denote the image $I(l)$ and the
visibility data (the measurements in the {\it uv}-plane) as $V(u)$.
Thus,
\begin{align}
I(l) &= \int^{\infty}_{-\infty}V(u)\exp^{-2\pi iul}du,\\
V(u) &= \int^{\infty}_{-\infty}I(l)\exp^{2\pi iul}dl,
\end{align}
where $l$ is the direction cosine corresponding to the Fourier
transform of the baseline co-ordinate $u$.

A point source in the image plane can be represented as a
$\delta$-function, $A_0\delta(l-l_0)$.  The corresponding visibility is
then the Fourier transform of a $\delta$-function,
\begin{equation}
\begin{split}
V(u) &= \int^{\infty}_{-\infty}I(l)\exp^{2\pi iul}dl\\
&= \int^{\infty}_{-\infty}\delta A_0(l-l_0) \exp^{2\pi iul}dl\\
&= A_0 \exp^{2\pi iul_0}\\
&= A_0[\cos(2\pi ul_0) + i\sin(2\pi ul_0)].
\end{split}
\end{equation}
This is a single-frequency sinusoid, the imaginary part lagging the
real part by $\pi/2$, and with a frequency depending on the position
of the source in the image plane.  For a source at the phase centre
($l=0$), the visibility is a source of constant amplitude, $V(u)=A_0$.

If we can approximate the image as a sum of point sources,
\begin{equation}
I(u) = \sum_{j=1}^{n} A_j \delta(l-l_j),
\end{equation}
then because of the linearity property of Fourier transforms, the
resulting visibility is then a sum of complex exponentials, all of
different frequencies,
\begin{equation}
\begin{split}
V(u) &= \sum_{j=1}^{n} A_j \exp^{2\pi iul_j}\\
&= V_{\rm Re}(u) + iV_{\rm Im}(u)\\
&= \sum_{j=1}^{n} \left[A_j\cos(2\pi ul_j) + i\sin[A_j\sin(2\pi ul_j)\right].
\label{eq:source_vis}
\end{split}
\end{equation}

\subsection{Discrete sampling}
Unfortunately, we do not fully sample the $u$-co-ordinate from
$-\infty$ to $\infty$.  Thus we introduce the sampling function
$S(u)$.  In principle, with known baseline lengths in the three
spatial co-ordinates ($L_X$, $L_Y$ and $L_Z$), then for a given hour
angle and declination of the phase reference centre ($H_0$ and
$\delta_0$ respectively), and a given observing wavelength, $\lambda$,
we can calculate the {\it uv}-coverage.
\begin{gather}
\begin{pmatrix}u \\ v \\ w\end{pmatrix} \quad = \quad \frac{1}{\lambda}
\begin{pmatrix}\sin H_0 & \cos H_0 & 0 \\ -\sin \delta_0\cos H_0 &
  \sin\delta_0 \sin H_0 & \cos\delta_0 \\ \cos\delta_0\cos H_0 &
  -\cos\delta_0\sin H_0 & \sin\delta_0\end{pmatrix} \quad
\begin{pmatrix}L_X \\ L_Y \\ L_Z\end{pmatrix}
\end{gather}

The sampling function $S(u)$ is defined to be 1 at each point in the
{\it uv}-plane where a measurement is made, and zero elsewhere.
Between two observations taken at times $t_1$ and $t_2$, the sampling
function will change from $S_1(u)$ to $S_2(u)$ as the hour angle
changes.  Denoting the Fourier transform of the sampling function
$S(u)$ as $B(l)$, we have
\begin{align}
V^m_1(u) &= S_1(u) V_1(u) \rightleftharpoons I^D_1(l) = I_1(l) \ast
B_1(l)\\
V^m_2(u) &= S_2(u) V_2(u) \rightleftharpoons I^D_2(l) = I_2(l) \ast
B_2(l),
\end{align}
where we have used $\ast$ to denote convolution and
$\rightleftharpoons$ to denote a Fourier transform.  $V$ is a true set
of visibilities, $V^m$ a measured set of visibilities, $I^D$ is a
dirty image, and $I$ the true sky brightness.

It is also worth noting that each baseline gives us two different
measurements, one at co-ordinate $u$ and the other at $-u$, and the
visibility measured at $-u$ is the complex conjugate of that measured
at $u$, for
\begin{equation}
V_{\nu}(\mathbf{r_1,r_2}) = \int I_{\nu}(\mathbf{s})\exp^{-2\pi i \nu
  \mathbf{s}\cdot(\mathbf{r_1-r_2})/c}\,d\Omega
\end{equation}
and
\begin{equation}
V_{\nu}(\mathbf{r_2,r_1}) = \int I_{\nu}(\mathbf{s})\exp^{-2\pi i \nu
  \mathbf{s}\cdot(\mathbf{r_2-r_1})/c}\,d\Omega
\end{equation}
and since the distribution of sky brightness, $I_{\nu}(\mathbf{s})$ is
the same, then $V_{\nu}(\mathbf{r_2,r_1}) =
V_{\nu}^{\ast}(\mathbf{r_1,r_2})$.  This means that the sky brightness
is always real.

\subsection{Practicalities}
Thus we can generate a random set of baselines $(L_X,L_Y,L_Z)$ and
generate their {\it uv}-coverage at two different hour angles,
\begin{equation}
u = \lambda^{-1}(L_X\sin H_0 + L_Y \cos H_0),
\end{equation}
to simulate how the {\it uv}-coverage changes with time.  The LOFAR
core will have a maximum baseline of $\sim 3000$\,m, a minimum
baseline of $\gtrsim 85$\,m (the station size), and 32 stations giving
$N(N-1)/2 = 496$ baselines.

For a source at sky position $(\alpha,\delta)$, and a phase reference
position $(\alpha_0,\delta_0)$, we can calculate its $(l,m)$
co-ordinates in the image as
\begin{align}
l &= \cos\delta\sin\Delta\alpha\\
m &= \sin\delta\cos\delta_0 - \cos\delta\sin\delta_0\cos\Delta\alpha,
\end{align}
where $\Delta\alpha = (\alpha-\alpha_0)$.

\section{Subtraction}
If we wish to detect transients, we need to compare two observations.
Comparing $I_1^D$ and $I_2^D$ will cause problems in detecting faint
transients, since the sidelobes of bright sources that have not been
deconvolved from the respective beams $B_1$ and $B_2$ will be present
in the map, and since the sampling functions will not be identical,
they will not subtract properly.  Thus in order to difference two
images, the images must first have been deconvolved.  Thus we turn to
subtraction in the {\it uv}-plane.  Since $S_1$ and $S_2$ are
different, we cannot directly subtract one set of visibilities from
another without interpolation, which requires {\it a priori} knowledge
of the true sky brightness distribution.  However, let us form the
summed set of visibilities 
\begin{equation}
V^m_{12}(u) = V^m_1(u) + (-1)\times V^m_2(u),
\end{equation}
which has a sampling function $S_{12}(u) = S_1(u)+S_2(u)$.

For a single point source, the measured visibilities, $V^m_{12}(u)$,
are then a pair of sinusoids, with identical frequencies and
amplitudes, but one positive and the other negative (or alternatively,
shifted by $\pi$ in phase), and sampled at different points.  But
since the Fourier transform is linear, these will each produce in the
dirty image $I^D_{12}(l)$ a point spread function (a $\delta$ function
convolved with the appropriate dirty beam) at the same position (owing
to the identical frequencies of the sinusoids) but with opposite
amplitudes.  Thus {\it they will sum to zero at the source position}.
The sidelobes will not cancel, since the dirty beams $B_1(l)$ and
$B_2(l)$ are different, so the noise in the image will be the flux
density of the brightest source in either dirty image multiplied by
the peak sidelobe intensity.

Now consider the effect of a transient in the second set of
visibilities.  There will be no positive-amplitude sinusoid to cancel
it out in the other set of visibilities.  Thus it will appear as a
negative component in the dirty map $I^D_{12}(l)$.  But since it only
appears in one of the two maps, its brightness will appear as half
that of its true brightness.  This can more easily be seen by
considering the discrete representation of the Fourier transform.  For
a set of complex visibilities $V_k$, measured at $2N$ points over $N$
baselines, $u_k$, the dirty spectrum can be computed as
\begin{equation}
I^D_{12}(l_j) = \frac{1}{2N}\sum_{k=1}^N (V_k \exp^{-2\pi i l_j u_k}+V_k^{\ast} \exp^{2\pi i l_j u_k}).
\end{equation}
The $1/2N$ term divides the sum by the total number of visibilities in
the two sets of measurements, $2N_1+2N_2$, whereas the transient is
only measured in set $N_2$.  Thus the amplitude of the transient is
reduced by a factor $N_2/(N_1+N_2)$.

\subsection{Caveats}
We note that this will {\bf only} work if the image consists solely of
point sources, such that the source visibility is the same regardless
of {\it uv}-coverage and a given frequency exponential identifies a
given source.  Otherwise, the visibilities corresponding to a single
source (Equation \ref{eq:source_vis}) contain several exponentials all
mixed in together, some of which are sampled only by certain
baselines, so will not necessarily cancel between two sets of
visibilities.  The cancellation should work as long as there are
sufficient visibility samples to distinguish all the exponentials in
the sum in both cases, and is not dependent on the source position in
the image.

\section{Examples}
Two examples are shown below.  In the first case (Fig.~1), $V_1(u)$ and
$V_2(u)$ are identical, but sampled differently (the hour angles
differ by $\pi/12$, i.e. one hour of time.  Each consists of a single
point source of amplitude 5.  The summed visibility set $V^m_{12}(u)$
is shown, as is the resulting dirty image, $I^D_{12}(l)$.  In the
second example (Fig.~2), a transient of amplitude 4 has been inserted into the
second set of visibilities.

\begin{figure}
\begin{center}
\subfigure[{\it uv}-data.  {\it Left}: the two individual data sets,
  $V_1^m(u)$ and $V_2^m(u)$.  {\it Right}: the summed data set,
  $V_{12}^m(u)$.]{\epsfig{figure=uv_data1.eps,width=\textwidth}}\\
\subfigure[Dirty images.  {\it Left}: individual dirty images,
  $I^D_1(l)$ and $I^D_2(l)$.  {\it Right}: dirty image of the summed
  data, $I^D_{12}(l)$.]{\epsfig{figure=skydata1.eps,width=\textwidth}}
\caption{One source in both {\it uv}-data sets.}
\end{center}
\end{figure}

\begin{figure}
\begin{center}
\subfigure[{\it uv}-data.  {\it Left}: the two individual data sets,
  $V_1^m(u)$ and $V_2^m(u)$.  {\it Right}: the summed data set,
  $V_{12}^m(u)$.]{\epsfig{figure=uv_data2.eps,width=\textwidth}}\\
\subfigure[Dirty images.  {\it Left}: individual dirty images,
  $I^D_1(l)$ and $I^D_2(l)$.  {\it Right}: dirty image of the summed
  data, $I^D_{12}(l)$.]{\epsfig{figure=skydata2.eps,width=\textwidth}}
\caption{One source in both {\it uv}-data sets, and a transient which
  appears only in the second.  The common source disappears and the
  transient pops up at half the brightness.}
\end{center}
\end{figure}

\end{document}
