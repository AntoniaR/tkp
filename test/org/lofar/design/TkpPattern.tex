\documentstyle[11pt]{article}
\textwidth 16cm
\hoffset -1.5cm
\textheight 23cm
\voffset -1.5cm
\setlength{\parindent}{0pt}
\setlength{\parskip}{\baselineskip}
\newcommand{\planitem}[3]{
\begin{tabular}{l@{\hspace{1cm}}p{12cm}}
  {\sl #1} & {\sl #2} \\  
           & #3 \\
\end{tabular}
}

\newenvironment{simplist}{
   \vspace*{-\baselineskip}
   \begin{itemize}
      \setlength{\itemsep}{-\parsep}
      \setlength{\parsep}{0pt}
}{
   \end{itemize}
   \vspace*{-\baselineskip}
}
  
\begin{document}
\begin{center}
   {\bf Design Priciples}\\[2em]
\end{center}

\bigskip\bigskip

\planitem{1}
{Identify the aspects of your application that vary and separate them from what stays the same}
{
     In our case changes will occur:

	\begin{simplist}
	\item at the interface level
	\item at the storage level
	\end{simplist}
}

\planitem{2}
{Program to an interface/supertype, not an implementation}
{
     In our case this implies:

	\begin{simplist}
	\item use an interface to represent each behavior
	\item each implementation of a {\i behavior} will implement one of those interfaces
	\end{simplist}
}

\planitem{3}
{Favor composition over inheritance}
{
     In our case this implies:

	\begin{simplist}
	\item Composition is: putting classes together via the HAS-A relationship
	\item this lets you change a behavior at runtime
	\end{simplist}
}

\end{document}
